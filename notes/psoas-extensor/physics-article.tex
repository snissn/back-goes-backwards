\documentclass[11pt]{article}

\usepackage{amsmath,amssymb,amsthm}
\usepackage{geometry}
\geometry{margin=1in}

\title{The Helical Extension Theorem:\\
Psoas--Ligament Coupling and Rotational-to-Axial Work Conversion in Hip Hyperextension}
\author{Concept Note}
\date{}

\newtheorem{theorem}{Helical Extension Theorem}
\newtheorem{definition}{Definition}
\newtheorem{remark}{Remark}
\newtheorem{assumption}{Assumption}

\begin{document}

\maketitle

\begin{abstract}
Standard descriptions of the iliopsoas emphasize its role as a hip flexor and trunk stabilizer,
and the hip joint is often modeled as a simple ball-and-socket or hinge for teaching purposes.
In extreme ranges of motion, however (e.g.\ deep backbends such as Urdhva Dhanurasana, martial
front splits, or maximal sprint extension), the anterior hip capsule and spiral ligaments become
fully tensioned. In this regime, the hip behaves like a screw constrained within a helical capsule.
We propose a simple kinematic and energetic model showing how an external rotation torque driven
by psoas can indirectly contribute to extension work through psoas--ligament coupling, without
requiring the psoas to change sign from flexor to extensor.
\end{abstract}

\section{Introduction}

Anatomy texts commonly describe psoas major as a powerful hip flexor that may also contribute
to external rotation and trunk control. The hip joint itself is a synovial ball-and-socket
joint reinforced by strong capsular ligaments, notably the iliofemoral (Y) ligament.
These ligaments are arranged in a spiral fashion from pelvis to proximal femur and tighten
in hip extension, with additional tightening reported in combined extension and external
rotation (a ``screw-home'' mechanism).

In functional movement disciplines such as yoga and martial arts, practitioners observe that
certain spiral activation patterns around the hip seem to improve lift, stability, and
force transmission in extreme positions (e.g.\ Wheel pose or loaded stances), even though
psoas is typically classified as an antagonist to hip extension. This note proposes a
mechanical explanation: in a ``helical constraint regime'' near terminal extension, rotation
and extension are kinematically coupled by ligament geometry, so that external rotation work
can manifest as effective extension work at the system level.

\section{Anatomical and Mechanical Background}

\subsection{Rigid bodies and constraints}

We model the system with three primary rigid bodies:

\begin{itemize}
  \item the pelvis and trunk (treated as a single body),
  \item the femur,
  \item the ground (feet fixed; closed kinetic chain).
\end{itemize}

The hip joint is modeled as a ball-and-socket with additional passive constraints arising from:

\begin{itemize}
  \item the iliofemoral, pubofemoral, and ischiofemoral ligaments,
  \item the anterior joint capsule and zona orbicularis.
\end{itemize}

These structures are known to run in a spiral from the acetabulum/ilium to the proximal femur
and to tighten in hip extension, with further tightening in some patterns of combined
extension and rotation.

\subsection{Psoas major}

Psoas major originates from T12--L5 vertebral bodies and transverse processes and inserts
on the lesser trochanter of the femur. It is classically described as:

\begin{itemize}
  \item a strong hip flexor,
  \item a contributor to lumbar flexion and lateral flexion,
  \item a potential external (lateral) rotator of the hip near neutral,
  \item an important postural stabilizer of the lumbopelvic region.
\end{itemize}

More detailed modeling shows that the rotational moment arm of psoas may vary with hip angle,
but for the purposes of this concept note we assume that in the extended or near-neutral
positions relevant to deep backbending, psoas can generate a net external rotation torque
about the hip.

\section{Kinematic Model of the Helical Constraint}

\begin{definition}[Joint coordinates]
Let
\begin{itemize}
  \item $\theta$ denote the hip extension angle measured from neutral standing
        ($\theta = 0^\circ$ in neutral, $\theta > 0^\circ$ in extension/hyperextension),
  \item $\phi$ denote the external rotation angle of the femur at the hip
        ($\phi > 0$ for external rotation),
  \item $L(\theta,\phi)$ denote an effective length of the anterior capsular/ligamentous
        complex between pelvis and femur.
\end{itemize}
\end{definition}

\begin{assumption}[Helical constraint regime]
There exists a range of motion near terminal extension in which:
\begin{enumerate}
  \item the anterior capsular ligaments (including the iliofemoral ligament) are fully
        tensioned and act as a primary constraint on motion,
  \item further motion of the hip approximately preserves $L$, so that
        \[
          L(\theta,\phi) \approx L_0,
        \]
        with $L_0$ constant over small variations of $(\theta,\phi)$,
  \item $L$ is differentiable in $\theta$ and $\phi$.
\end{enumerate}
\end{assumption}

Differentiating the constraint $L(\theta,\phi) = L_0$ yields
\begin{equation}
  \frac{\partial L}{\partial \theta}\, d\theta
  + \frac{\partial L}{\partial \phi}\, d\phi = 0.
\end{equation}

Provided $\partial L / \partial \theta \neq 0$ in this regime, we obtain the local coupling
relation
\begin{equation}
  \frac{d\theta}{d\phi} = - \frac{\partial L / \partial \phi}{\partial L / \partial \theta}
  =: \lambda,
\end{equation}
where $\lambda$ is an effective helical pitch that characterizes the coupling between
external rotation and extension near the chosen configuration.

\begin{assumption}[Sign of the helical pitch]
Empirically, the spiral orientation of the capsular ligaments and their reported tightening
in combined extension and external rotation suggest that in the relevant end-range regime,
a small increase in external rotation is associated with a small increase in extension.
We therefore assume $\lambda > 0$ locally.
\end{assumption}

Under these assumptions, small changes obey
\begin{equation}
  \Delta \theta \approx \lambda\, \Delta \phi
\end{equation}
for sufficiently small variations around a fixed extended configuration.

\section{Energetic Contribution of Psoas Rotation Torque}

Let $\tau_{\phi,\text{psoas}}$ denote the torque about the hip's rotational axis generated
by psoas in the external rotation direction.

The incremental work done by psoas in the rotational coordinate is
\begin{equation}
  dW_{\text{psoas}} = \tau_{\phi,\text{psoas}}\, d\phi.
\end{equation}

Using the kinematic coupling $d\theta = \lambda\, d\phi$, we may write
\begin{equation}
  d\phi = \frac{1}{\lambda}\, d\theta,
\end{equation}
and therefore
\begin{equation}
  dW_{\text{psoas}} = \tau_{\phi,\text{psoas}}\, \frac{1}{\lambda}\, d\theta.
\end{equation}

We define the \emph{effective extension work} of psoas in the helical regime as
\begin{equation}
  W_{\text{ext, eff}}
  := \int_{\theta_1}^{\theta_2} \tau_{\phi,\text{psoas}}(\theta)\, \frac{1}{\lambda(\theta)}\, d\theta.
\end{equation}

Note that this does \emph{not} imply that psoas has become a mechanical hip extensor
in the sagittal plane. Its classical flexion moment arm remains; rather, the geometry
of the capsular constraint causes some of the rotational work it performs to appear
as axial motion in the extension coordinate.

\begin{theorem}[Helical Extension Theorem]
Assume:
\begin{enumerate}
  \item The hip is in a configuration where the anterior capsular ligaments are fully
        tensioned and impose a differentiable helical constraint $L(\theta,\phi) = L_0$.
  \item The local helical pitch $\lambda = d\theta / d\phi$ is nonzero and positive.
  \item Psoas generates a nonzero external rotation torque $\tau_{\phi,\text{psoas}}$
        about the hip.
\end{enumerate}
Then, for any motion respecting the constraint between $\theta_1$ and $\theta_2$ in this regime,
the effective extension work of psoas satisfies
\[
  W_{\text{ext, eff}} =
  \int_{\theta_1}^{\theta_2} \tau_{\phi,\text{psoas}}(\theta)\, \frac{1}{\lambda(\theta)}\, d\theta
  \neq 0
\]
whenever $\tau_{\phi,\text{psoas}}$ is not identically zero.

In a closed kinetic chain (feet fixed on the ground), this extension work corresponds to
a forward and upward displacement of the pelvis relative to the feet, so that psoas can
indirectly assist the global extension of the body in configurations such as deep backbends.
\end{theorem}

\begin{remark}[Compatibility with classical anatomy]
The theorem does not require psoas to change sign from hip flexor to hip extensor. Locally,
its flexion moment arm remains, and net extension requires the dominant action of musculature
such as gluteus maximus and hamstrings. The result simply states that in a constrained
helical regime, rotational work performed by psoas can contribute indirectly to the
extension degree of freedom.
\end{remark}

\begin{remark}[Practical interpretations]
In yoga backbends such as Urdhva Dhanurasana, practitioners often report that subtle
spiral actions of the femurs and a sense of ``drawing into the hip sockets'' create
more lift and less compression. In martial arts, ``screwing'' the legs into the ground
to load the kua is a similar experiential description. The helical extension model
provides a mechanical language for these observations: psoas supplies a controlled
external rotation torque that engages the spiral ligaments, and the resulting geometric
coupling improves extension support at the system level.
\end{remark}

\section{Limitations and Future Work}

This concept note is intentionally simplified. Among the limitations:

\begin{itemize}
  \item The effective ligament length $L(\theta,\phi)$ and helical pitch $\lambda(\theta)$
        have not been quantitatively measured in vivo for the extreme positions considered.
  \item The rotational moment arm of psoas may vary substantially across individuals
        and hip flexion/extension angles.
  \item Other muscles (e.g.\ deep rotators, gluteus maximus) also contribute to external
        rotation torque and thus may share or dominate the proposed effect.
  \item The model neglects three-dimensional translations of the femoral head and the
        complex anisotropic behavior of ligaments and capsule.
\end{itemize}

Nevertheless, the helical extension theorem offers a useful conceptual bridge between
standard anatomy, advanced asana and martial practice, and formal mechanics. It motivates
further imaging, modeling, and experimental work to quantify capsular and ligamentous
constraints at end-range hip motion.

\end{document}

